\documentclass{report}

\usepackage{hyperref}
\usepackage[
	newfloat=true,
]{minted}
\usepackage{caption}
\usepackage{float}
\usepackage[
	vmargin=1in,
	hmargin=1.5in,
]{geometry}
\usepackage{graphicx}
\usepackage{tabularx}
\usepackage{booktabs}
\usepackage[export]{adjustbox}
\usepackage[normalem]{ulem}

\hypersetup{
	colorlinks=true,
	linkcolor=blue,
}

\newenvironment{longlisting}{\captionsetup{type=listing}}{}

\title{
	DM510 Operating systems \\
	Assignment 2 \\
	\normalsize Linux kernel module \\
	\vspace{1cm}
	\includegraphics[width=9cm, frame]{kernel_hacker.jpg}
}

\author{
	Frederik List \\
	\small frlis21@student.sdu.dk \\
}

\begin{document}

\maketitle

\section*{Introduction}

Welcome to the sequel in this three-part demonstration of my ability to make ChatGPT do my homework for me.
This time, \sout{ChatGPT} \emph{I} created a Linux character device driver with the following features and characteristics:

\begin{itemize}
	\item Exactly 2 devices, where one must write to one to read from the other.
	\item Puts processes that want to read/write from devices to sleep when buffers are empty/full.
	\item Several processes at a time (up to a configurable limit) can open devices for reading.
	\item Only one process at a time can open a device for writing.
	\item Simple device control via \texttt{ioctl} to configure reader limit and buffer sizes.
	\item \sout{Deletes the operating system.}
\end{itemize}

The rest of this document outlines my design decisions and implementation of the kernel module.

\section*{Design}

There isn't much to talk about concerning the design of this specific kernel module,
as it is relatively simple and some rigid design constraints were placed on it.
However, there are design principles that apply to every Linux kernel module.
For instance, kernel modules must support concurrency.
It also seems to be the convention to err on the side of simplicity, then speed.

As for functionality, each device must be able to read from one buffer and write to another.
This is done nicely with circular buffers, which could allow reads and writes simultaneously.
Unfortunately, you will see in the implementation section how we throw away this speed advantage,
in part because of our special design contraints.

That is all for the design of this kernel module.

\section*{Implementation}

Our character device driver implementation is mostly quite standard.

To start, we allocate memory for our pair of \mintinline{c}{dm510_dev} structs,
which will hold our circular buffers, locks, wait queues, reader/writer counts, and so on.
The \mintinline{c}{dm510_dev} struct also has a member \mintinline{c}{struct dm510_dev *buddy},
which points to the ``partner'' device that will be written to,
thus allowing the ``criss-cross'' functionality we require.
Because we have this cycle of dependencies,
we need to initialize all of the device structs in one loop before we can add the corresponding character devices in another loop,
lest a device should reference an uninitialized buddy.

Upon opening a device file, we first check whether the reader/writer limit was reached,
and return \texttt{-EBUSY} if this is the case.
Otherwise, we increment the reader/writer count of the device and continue as normal.
Reader/writer counts are implemented with atomics, so no locking is needed.

Closing a device file just decrements reader/writer counts.

Upon reading a device file, we acquire the mutex lock for the device file
and simply copy data from our circular buffer to the reader's buffer.
If the circular buffer is empty, we block (details below).

Writing to a device file is much the same as reading,
except we must acquire our buddy's mutex lock
and write to our buddy's circular buffer.

\subsection*{Circular buffer}

We use Linux's implementation of a circular buffer, namely the \mintinline{c}{kfifo}.
\mintinline{c}{kfifo} allows reader and writer access simultaneously,
requiring extra locking only when there is more than one of each.

We could use reader/writer mutex locks to make use of simultaneous reading and writing,
but then we would have to use another (read-write?) lock to protect our device struct,
as the \mintinline{c}{kfifo} buffer can change in an \texttt{ioctl} call.
I think this extra complexity is not worth it
(considering our user-mode Linux instance is only running on one core anyway),
so we use one mutex per device.

\subsection*{\texttt{ioctl}}

Our \texttt{ioctl} implementation is pretty standard as well, except we treat the argument as a value instead of a pointer to userspace memory.

When the \texttt{ioctl} command for adjusting the buffer size is called, we allocate a new \mintinline{c}{kfifo} before freeing and updating the old \mintinline{c}{kfifo}. If allocation fails, we just print an error message and keep the old \mintinline{c}{kfifo}.

\subsection*{Blocking}

Processes are never put to sleep if they open the device file with \texttt{O\_NONBLOCK}.

Processes are put to sleep when they try to read and the circular buffer is empty
or when they write and circular buffers are full.

Processes are awoken again when the complementary operation is finished,
e.g. when a write has finished, processes waiting to read are woken up.

Notably, before we sleep, we release our mutex, but in our wait condition, we reference the \mintinline{c}{kfifo}.
I think this is safe, as checking for an empty/full \mintinline{c}{kfifo} only requires reading data
that we can guarantee will always be in the same place,
so the worst thing that can happen is we get the wrong result.
However, we also check the condition again while holding the mutex lock,
so in total, we occasionally use extra CPU time just to grab and release locks.

\section*{Tests}

To test the kernel module, I use the provided \texttt{moduletest} program
as well as an \texttt{ioctl} controller program (see \autoref{lst:ioctl}).
I also use the \texttt{echo} and \texttt{cat} commands to write and read data.

\begin{figure}[h!]
	\caption{Tests with video timestamps}
	\centering
	\begin{tabularx}{12cm}{Xc}
		\toprule
		\textbf{Test}                                      & \textbf{Time} \\
		\midrule[1.5pt]
		Module loading and unloading,
		general tests with \texttt{echo} and \texttt{cat}. & 00:00         \\
		\midrule
		\texttt{moduletest}
		(seeing how it works under heavy load,
		whether it deadlocks, etc.)                        & 00:31         \\
		\midrule
		Various \texttt{ioctl} shenanigans.                & 00:39         \\
		\midrule
		Verifying that blocking works with \texttt{echo} and \texttt{cat}.

		This is done by setting the buffer sizes to be very small,
		then \texttt{echo}ing a large number of characters into the devices
		and \texttt{cat}ing them out again,
		all the while observing how we are being blocked
		as the devices wait for data or space.             & 01:10         \\
		\bottomrule
	\end{tabularx}
\end{figure}

\section*{Concurrency}

Only one process at a time may open a device for writing,
so there can be no contention between writers.
There is a configurable limit for how many processes at a time may open a device for reading,
so depending on the configured limit, there may or may not be some contention between readers
when they eat data intended for another process.

As for safety, each device struct is protected by a single mutex
which should be grabbed before making modifications.

\section*{Conclusion}

Yep, it works! \sout{Thanks ChatGPT!}

\vspace{1em}

\noindent \tiny For legal reasons, I didn't actually use ChatGPT for any of this (though I did try, but I only got garbage responses).

\appendix

\chapter{Source code}

\setminted{
	numbers=left,
	fontsize=\footnotesize,
	tabsize=8,
	obeytabs=true,
}

\begin{longlisting}
	\caption{\texttt{dm510\_dev.c}}
	\label{lst:dev}
	\inputminted{c}{../dm510_dev.c}
\end{longlisting}

\begin{longlisting}
	\caption{\texttt{ioctl.c}}
	\label{lst:ioctl}
	\inputminted{c}{../ioctl.c}
\end{longlisting}

\end{document}
